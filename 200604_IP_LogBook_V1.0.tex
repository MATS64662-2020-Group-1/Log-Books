\documentclass{article}
\usepackage{xcolor}


\title{Image Processing – Methodology Log}
\author{Joshua Collins}

\begin{document}
\maketitle

\section{Image Processing Logs}

See below the steps for image processing:

\subsection{Tested ‘Skimage’ - python image processing package (link: https://scikit-image.org/)  \textcolor{blue}{(17/04/2020)}}

\begin{itemize}
\item \textcolor{red}{See Image-Processing/testcase\_IP\_V1.0.ipynb}
\item Tested algorithms on an image of a ferritic microstructure taken from Materials Science and Engineering: An Introduction, W. D. Callister, 6th Edition.
\item Testing included:
\begin{itemize}
\item Manipulating image colour and contrast
\item Using filters (e.g. sobel filter) and transforms (e.g. watershed)
\item Combining manipulation, filters and transforms to segment the image into separate labelled grains
\item Use skimage measuring tools to measure properties of segmented grains (i.e. measured areas and used this to estimate and average grain size)
\end{itemize}
\item Testcase 1 primarily used to test skimage algorithms and understand the image manipulation interface
\item Re-named to 'testcase\_IP\_V1.0' on 23/04/2020
\end{itemize}

\subsection{Tested ‘Skimage’ processing and subsequent measuring tools on a previously analysed microstructure \textcolor{blue}{(21/04/2020)}}

\begin{itemize}
\item \textcolor{red}{See Image-Processing/testcase\_IP\_V2.0.ipynb}
\item Tested algorithms on a thermally etched steel micrograph showing prior austenite grains (image was taken by Josh and approved by Dr Ed Pickering [lead research supervisor])
\item Linear intercept method previously used to determine grain size of 24.9 microns
\item Test case was split into various steps (\textcolor{red}{see notebook}):
\begin{itemize}
\item Importing Packages
\begin{itemize}
\item Imported various skimage sub-packages as well as os, numpy, and matplotlib.pyplot
\end{itemize}
\item Reading the Image
\begin{itemize}
\item Image was read, type and shape was determined before manipulation
\end{itemize}
\item Manipulating the Image
\begin{itemize}
\item Image cropped to region of highest focus
\end{itemize}
\item Segmenting the Image
\begin{itemize}
\item Used a combination of a Sobel filter and a watershed transform to segment the image and label each grain
\end{itemize}
\item Measuring Grain Size
\begin{itemize}
\item Skimages’ ‘measure’ was used to determine average grain size (assuming all grains were perfect circles)
\item Pixels were ceonverted to microns using the scale bar on the image
\item Average grain size found
\end{itemize}
\end{itemize}
\item Measured grain size = 32.7 microns
\begin{itemize}
\item Discrepancy due to technical assumptions (i.e. grains = circles) and overestimation of grain boundary width during segmentation
\end{itemize}
\end{itemize}

\subsection{Edited testcase 2 \textcolor{blue}{(23/04/2020)}}

\begin{itemize}
\item \textcolor{red}{See Image-Processing/testcase\_IP\_V2.0.ipynb history log}
\item Edited section 3. Manipulating the Image
\begin{itemize}
\item Adjusted image exposure to help differentiate between grain boundary and grain interior
\end{itemize}
\item Re-ran rest of notebook
\item Change resulted in a new measured grain size = 23.5 microns 
\begin{itemize}
\item  approx. 1.4 microns difference between experimentally determined value
\end{itemize}
\end{itemize}

\subsection{Edited testcase 2 \textcolor{blue}{(28/06/2020)}}

\begin{itemize}
\item \textcolor{red}{See Image-Processing/testcase\_IP\_V2.0.ipynb history log}
\item To produce a binary image of the segmented grains
\item Determined that the best place to obtain this image is before labelling 
\begin{itemize}
\item When the image is named ‘segmentation’ (after a watershed transform is applied)
\end{itemize}
\item Binary image required for meshing using Gmsh software
\item Conclusions: happy with image processing notebook to successfully process images for meshing and subsequent FEM
\begin{itemize}
\item Possible automation of image processing could be coded in python
\end{itemize}
\end{itemize}

\subsection{Used image processing notebook to produce binary image of Ti64 microstructure and then meshed binary image using Gmesh software \textcolor{blue}{(01/06/2020)}}

\begin{itemize}
\item \textcolor{red}{See Image-Processing/testcase\_IP\_V3.0.ipynb, Ti64\_binary\_V1.0.png \& Ti64\_V1.0.mesh file}
\item Ti64 image was obtained from Pratheek with his consent
\item Uses skimage packages previously mentioned to produce a binary image of segmented grains and saves to repository (image called: Ti64\_binary\_V1.0.png)
\item Exported image into Gmsh software (https://gmsh.info/) to create a square-based mesh of image for future FEM analysis (mesh called: Ti64\_V1.0.mesh)
\end{itemize}

\subsection{Used image processing notebook to produce binary image of a single grain in a Ti64 microstructure and then meshed binary image using Gmesh software \textcolor{blue}{(01/06/2020)}}

\begin{itemize}
\item \textcolor{red}{See Image-Processing/testcase\_IP\_V4.0.ipynb, Ti64\_binary\_V2.0.png \& Ti64\_V2.0.mesh file}
\item Ti64 image was obtained from Pratheek with his consent
\item Additional skimage sub-packages used to isolate a single grain
\begin{itemize}
\item Used ‘threshold’ filters and ‘clean\_border’ segmentation to remove surrounding grains from the image
\end{itemize}
\item Image saved to repository as binary image (named: Ti64\_binary\_V2.0.png)
\item Used Gmsh to create square-bsaed mesh of image for FEM (mesh called: Ti64\_V2.0.mesh) 
\item Conclusions: happy with image processing testcases (both image manipulation on python notebook and meshing in Gmsh). End testing.
\end{itemize}

\section{Current Work}

\subsection{Producing binary images for FEM}

\subsection{Automate python notebook}

\end{document}
